\documentclass[12pt]{article}
% \usepackage[utf8x]{inputenc}

% \usepackage{natbib}

\title{Ethical principles for the application of artificial intelligence (AI) in nuclear medicine}
\date{}
% \author{}

\begin{document}
\maketitle
\noindent
\textbf{\large Abstract}:\\ \\ \Artificial intelligence (AI) in Nuclear Medicine is a game-changing technology that has the potential to
revolutionize clinical and scientific practice. Although artificial intelligence (AI) is not new to nuclear medicine, medical imaging, or medicine in general, recent developments in machine learning (ML) and deep learning (DL)
have accelerated the adoption of novel algorithms. The task of identifying and resolving ethical dilemmas arises
as a result of this development. With the fast adoption of ML and DL in Nuclear Medicine, ethical challenges are
being recognized at the same time as innovation and deployment, yet contingencies and ethical underpinnings
for proactive management are behind. In clinical and research practice, ethical issues are divided into three
categories: data usage, algorithm selection, and deployment strategy. The ethical issues surrounding the use of
AI in Nuclear Medicine are examined in this paper, and 16 ethical principles are offered as a framework for
applying AI principles in Nuclear Medicine
\noindent
\\ \\ \textbf{\large Table of content:}
\\ \\1.Introduction \\ 2.Ethical challenge \\  3.Ethical principles in AI \\  4.Summary of ethical principles \\ 5.Conclusion \\6.Reference \\ \\ \textbf{\large Introduction:} \\ \\ Artificial intelligence (AI) in Nuclear Medicine is a game-changing technology that has the
potential to revolutionize clinical and scientific practice. Although artificial intelligence (AI) is not new to nuclear
medicine, medical imaging, or medicine in general, recent developments in machine learning (ML) and deep
learning (DL) have accelerated the adoption of novel algorithms. The task of identifying and resolving ethical
dilemmas arises as a result of this development. With the fast adoption of ML and DL in Nuclear Medicine, ethical challenges are being recognized at the same time as innovation and deployment, yet contingencies and
ethical underpinnings for proactive management are behind. In clinical and research practice, ethical issues are
divided into three categories: data usage, algorithm selection, and deployment strategy. Artificial intelligence (AI) is a broad phrase that refers to algorithms that are supposed to mimic certain
characteristics of intelligent human behavior, such as pattern recognition, problem solving, and reasoning. In
medical imaging, an artificial neural network (ANN) is an image processing system made up of layers of linked
nodes that imitate the neuronal connections of the human brain. ANNs are algorithms that analyze data and
identify trends or patterns that may be used to make predictions (e.g. classification of disease). A convolutional
neural network (CNN) is a deep learning ANN that uses a convolutional procedure to extract features from an
image, while an ANN usually takes feature data as input. Machine learning (ML) is a kind of AI that applies ML techniques without being explicitly programmed via data
analysis. After learning from human-defined instructional scenarios characteristic of an ANN, ML is often
connected with addressing logic issues. Deep learning (DL) is a sub-type of machine learning (ML) that uses a
number of processing layers (depth) to discover complicated patterns in images, similar to a CNN. Artificial
intelligence (AI) essentially imitates human intellect, but synthetic intelligence (SI) delivers real higher-order
reasoning utilizing technologies such as quantum logic. In general, AI has two sorts of presence in the patient
care experience in healthcare: virtual and physical. Virtual AI solutions are most widely used in Nuclear
Medicine. However, as the area develops, it will be necessary to tackle the ethical issues that come with
physically present AI solutions. AI in Nuclear Medicine and Molecular Imaging ushers in a new age of clinical and scientific capabilities that have
been reengineered and redesigned. AI has the ability to enhance workflow and efficiency while also lowering
costs, improving accuracy, and facilitating research and discovery. This entails a responsibility of care to patients
to ensure that AI-assisted diagnosis or therapy results in the best possible outcomes. The ethical issues that
occur when utilizing human data to construct person-targeted applications are perhaps the most challenging
problem to handle in AI application to Nuclear Medicine. These ethical concerns are divided into three categories: the data utilized, the algorithms employed, and the
way they are used in practice. These three areas in the interplay between ethical and social challenges for AI in
medical imaging are also identified in a white paper from the French radiology community and a joint
declaration from European and North American associations (Fig. 1)\\
